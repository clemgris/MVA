\documentclass[a4paper]{article}
\usepackage[english]{babel}
\usepackage[utf8]{inputenc}
\usepackage{fancyhdr}
\usepackage{hyperref}
\usepackage{amsmath,amsfonts,amssymb,amsthm}
\usepackage[a4paper, bottom=1.3in, top=1.3in, right=1in, left=1in]{geometry}
\usepackage[usenames,dvipsnames]{xcolor}
\usepackage[lined,boxed]{algorithm2e}
\usepackage{natbib}
\usepackage{dsfont}
\usepackage{tikz}
\usetikzlibrary{calc}
\definecolor{amaranth}{rgb}{0.9, 0.17, 0.31}
\newcommand{\rcol}[1]{{\color{amaranth}#1}}

\usepackage{todonotes}
\newcommand{\todomp}[1]{\todo[color=Green!10, inline]{\small MP: #1}}
\newcommand{\todompout}[1]{\todo[color=Green!10]{\scriptsize MP: #1}}

\newcommand{\wh}[1]{\widehat{#1}}
\newcommand{\wt}[1]{\widetilde{#1}}
\newcommand{\transp}{\intercal}

%%%%%%%%%%%%%%%%%%%%%%%%%%%%%%%%%%%%%%%%%%%
% Insert your name here
\newcommand{\fullname}{FILL fullname command at the beginning of latex document}
%%%%%%%%%%%%%%%%%%%%%%%%%%%%%%%%%%%%%%%%%%%

\newcommand{\lecture}[3]{
   \pagestyle{myheadings}
   \thispagestyle{plain}
   \newpage
   \setcounter{page}{1}
   \noindent
   \begin{center}
   \framebox{
      \vbox{\vspace{2mm}
              \hbox to .97\textwidth { {\bf MVA: Reinforcement Learning (2022/2023) \hfill Assignment 3} }
       \vspace{6mm}
       \hbox to .97\textwidth { {\Large \hfill #1 \hfill } }
       \vspace{6mm}
       \hbox to .97\textwidth { {Lecturers: \it M. Pirotta  \hfill {{\footnotesize(
           December 12, 2022
        %    \today
           )}}} }
      \vspace{2mm}}
   }
   \end{center}
   Solution by {\color{amaranth}\fullname}
   \markboth{#1}{#1}
   \vspace*{4mm}
}


\DeclareMathOperator*{\argmax}{\arg\,\max}
\DeclareMathOperator*{\argmin}{\arg\,\min}
\DeclareMathOperator*{\arginf}{\arg\,\inf}


\setlength{\parindent}{0cm}
\begin{document}
\lecture{Exploration in Reinforcement Learning (theory)}{3}


\pagestyle{fancy}
\fancyhf{}
\rhead{Full name: {\color{amaranth}\fullname}}
\lhead{Exploration in Reinforcement Learning}
\cfoot{\thepage}

\textbf{Instructions}
\begin{itemize}
    \item The deadline is \textbf{January 20, 2023. 23h59}
    \item By doing this homework you agree to the \emph{late day policy, collaboration and misconduct rules} reported on Piazza.
    \item \textbf{Mysterious or unsupported answers will not receive full credit}.
    A correct answer, unsupported by calculations, explanation, or algebraic work will receive no credit; an incorrect answer supported by substantially correct calculations and explanations might still receive partial credit.
    \item Answers should be provided in \textbf{English}.
\end{itemize}


% \section{UCB}
% Denote by $S_{j,t} = \sum_{k=1}^t X_{i_k,k} \cdot \mathds{1}(i_k = j)$ and by $N_{j,t} = \sum_{k=1}^t \mathds{1}(i_k = j)$ the cumulative reward and number of pulls of arm $j$ at time $t$. Denote by $\wh\mu_{j,t} = \frac{S_{j,t}}{N_{j,t}}$ the estimated mean. Recall that, at each timestep $t$, UCB plays the arm $i_t$ such that
% \[
%     i_t \in \argmax_{j} \wh\mu_{j,t} + U(N_{j,t}, \delta)
% \]
% Is $\wh\mu_{j,t}$ an unbiased estimator (i.e., $\mathbb{E}_{UCB}[\wh\mu_{j,t}] = \mu_j$)? Justify your answer.

\section{Best Arm Identification}
In best arm identification (BAI), the goal is to identify the best arm in as few samples as possible.
We will focus on the fixed-confidence setting where the goal is to identify the best arm with high probability $1-\delta$ in as few samples as possible.
A player is given $k$ arms with expected reward $\mu_i$. At each timestep $t$, the player selects an arm to pull ($I_t$), and they observe some reward ($X_{I_t,t}$) for that sample.
At any timestep, once the player is confident that they have identified the best arm, they may decide to stop.

\paragraph{$\delta$-correctness and fixed-confidence objective.}
Denote by $\tau_\delta$ the stopping time associated to the stopping rule, by $i^\star$ the best arm and by $\wh{i}$ an estimate of the best arm.
An algorithm is $\delta$-correct if it predicts the correct answer with probability at least $1-\delta$. Formally, if $\mathbb{P}_{\mu_1, \ldots, \mu_k}(\wh{i} \neq i^\star) \leq \delta$ and $\tau_{\delta} < \infty$ almost surely for any $\mu_1, \ldots, \mu_k$.
Our goal is to find a $\delta$-correct algorithm that minimizes the sample complexity, that is, $\mathbb{E}[\tau_\delta]$ the expected number of sample needed to predict an answer. Assume that the best arm $i^\star$ is \emph{unique} (i.e., there exists only one arm with maximum mean reward).

\vspace{.2in}
\underline{Notation}
\begin{itemize}
    \item $I_t$: the arm chosen at round $t$.
    \item $X_{i,t} \in [0,1]$: reward observed for arm $i$ at round $t$.
    \item $\mu_i$: the expected reward of arm $i$.
    \item $\mu^\star = \max_i \mu_i$.
    \item $\Delta_i = \mu^\star - \mu_i$: suboptimality gap.
\end{itemize}

Consider the following algorithm
\begin{algorithm}[h]
    \SetAlgoLined
    \DontPrintSemicolon
    \KwIn{$k$ arms, confidence $\delta$}
    $S = \{1, \ldots, k\}$\;
    \For{$t = 1, \ldots$}{
        Pull \textbf{all} arms in $S$\;
        $S = S \setminus \Big\{i \in S \;:\; \exists j \in S,\; \wh{\mu}_{j,t} - U(t,\delta') \geq \wh{\mu}_{i,t} + U(t, \delta')  \Big\}$\;
        \If{$|S|=1$}{
            STOP\;
            \textbf{return} $S$\;
        } 
    }
\end{algorithm}

The algorithm maintains an active set $S$ and an estimate of the empirical reward of each arm $\wh\mu_{i,t} = \frac{1}{t} \sum_{j=1}^t X_{i,j}$.
\begin{itemize}
    \item Compute the function $U(t,\delta)$ that satisfy the any-time confidence bound. Let 
    \[ 
        \mathcal{E} = \bigcup_{i=1}^{k}\bigcup_{t=1}^{\infty} \left\{ | \wh{\mu}_{i,t} - \mu_i | > U(t,\delta')\right\}.
    \]
    Using Hoeffding's inequality and union bounds, shows that $\mathbb{P}(\mathcal{E}) \leq \delta$ for a particular choice of $\delta'$. This is called ``bad event'' since it means that the confidence intervals do not hold.
    \item Show that with probability at least $1-\delta$, the optimal arm $i^\star =\argmax_i \{\mu_{i}\}$ remains in the active set $S$. Use your definition of $\delta'$ and start from the condition for arm elimination. From this, use the definition of $\neg \mathcal{E}$.
    \item Under event $\neg \mathcal{E}$, show that an arm $i \neq i^\star$ will be removed from the active set when $\Delta_i \geq C_1 U(t, \delta')$ for some constant $C_1 \in \mathbb{N}$. Compute the time required to have such condition for each non-optimal arm. Use the condition of arm elimination applied to arm $i^\star$.\footnote{Note that $at \geq \log(bt)$ can be solved using Lambert W function. We thus have $t \geq \frac{-W_{-1}(-a/b)}{a}$ since, given $a = \Delta^2_i$ and $b=2k/\delta$, $-a/b \in (-1/e,0)$. We can make the bound more explicit by noticing that $-1-\sqrt{2u}-u\leq W_{-1}(-e^{-u-1})\leq -1 - \sqrt{2u} - 2u/3$ for $u>0$~\citep{Chatzigeorgiou16}. Then $t \geq \frac{1+\sqrt{2u}+u}{a}$ with $u=\log(b/a) - 1$.}
    \item Compute a bound on the sample complexity (after how many \emph{pulls} the algorithm stops) for identifying the optimal arm w.p. $1-\delta$. 
    \item We assumed that the optimal arm $i^\star$ is unique. Would the algorithm still work if there exist multiple best arms? Why?
\end{itemize}

Note that also a variations of UCB are effective in pure exploration.


\section{Regret Minimization in RL}
Consider a finite-horizon MDP $M^\star = (S, A, p_h, r_h)$ with stage-dependent transitions and rewards. Assume rewards are bounded in $[0,1]$.
We want to prove a regret upper-bound for UCBVI. We will aim for the suboptimal regret bound ($T=KH$)
\[
    R(T) = \sum_{k=1}^K V^\star_1(s_{1,k}) - V^{\pi_k}_1(s_{1,k}) = \wt{O}(H^2S\sqrt{AK})
\]
Define the set of plausible MDPs as 
\[
    \mathcal{M}_k = \{ M = (S,A, p_{h,k}, r_{h,k}) ~:~ r_{h,k}(s,a) \in \beta^r_{h,k}(s,a), p_{h,k}(\cdot|s,a) \in \beta^p_{h,k}(s,a)  \}
\]
Confidence intervals can be anytime or not.

\begin{itemize}
    \item Define the event $\mathcal{E} = \{\forall k, M^\star \in \mathcal{M}_k\}$. Prove that $\mathbb{P}(\neg\mathcal{E}) \leq \delta/2$. First step, construct a confidence interval for rewards and transitions for each $(s,a)$ using Hoeffding and Weissmain inequality (see appendix), respectively. So, we want that
    \[
        \mathbb{P}\Big(\forall k,h,s,a : \wh{r}_{hk}(s,a) - r_h(s,a)| \leq \beta_{hk}^r(s,a) \wedge \|\wh{p}_{hk}(\cdot|s,a) - p_{h}(\cdot|s,a)\|_1\leq \beta_{hk}^p(s,a)\Big) \geq 1-\delta/2
    \]
    
    \item Define the bonus function and consider the Q-function computed at episode $k$
    \[
        Q_{h,k}(s,a) = \wh{r}_{h,k}(s,a) + b_{h,k}(s,a) + \sum_{s'} \wh{p}_{h,k}(s'|s,a) V_{h+1,k}(s')
    \]
    with $V_{h,k}(s) = \min\{H, \max_a Q_{h,k}(s,a)\}$. Recall that $V_{H+1,k}(s) = V_{H+1}^\star(s) = 0$.
    Prove that under event $\mathcal{E}$, $Q_k$ is optimistic, i.e.,
    \[
        Q_{h,k}(s,a) \geq Q^{\star}_h(s,a), \forall s,a
    \]
    where $Q^\star$ is the optimal Q-function of the unknown MDP $M^\star$.
    Note that $\wh{r}_{H,k}(s,a) + b_{H,k}(s,a) \geq r_{H,k}(s,a)$ and thus $Q_{H,k}(s,a) \geq Q^\star_H(s,a)$ (for a properly defined bonus). Then use induction to prove that this holds for all the stages $h$.

    \item In class we have seen that
    \begin{equation}
        \label{eq:ucbvi_rec}
        \delta_{1k}(s_{1,k}) \leq \sum_{h=1}^H Q_{hk}(s_{hk},a_{hk}) - r(s_{hk},a_{hk}) - \mathbb{E}_{Y\sim p(\cdot|s_{hk},a_{hk})}[V_{h+1,k}(Y)]) + m_{hk}
    \end{equation}
    where $\delta_{hk}(s)=V_{hk}(s) - V_h^{\pi_k}(s)$ and $m_{hk} = \mathbb{E}_{Y\sim p(\cdot|s_{hk},a_{hk})}[\delta_{h+1,k}(Y)] - \delta_{h+1,k}(s_{h+1,k})$.
    We now want to prove this result. Denote by $a_{hk}$ the action played by the algorithm (you will have to use the greedy property).
    \begin{enumerate}
        \item Show that $V^{\pi_{k}}_h(s_{hk}) = r(s_{hk},a_{hk}) + \mathbb{E}_{p}[V_{h+1,k}(s')] - \delta_{h+1,k}(s_{h+1,k}) - m_{h,k}$
        \item Show that $V_{h,k}(s_{hk}) \leq Q_{h,k}(s_{hk},a_{hk})$.
        \item Putting everything together prove Eq.~\ref{eq:ucbvi_rec}.
    \end{enumerate}

    \item Since $(m_{hk})_{hk}$ is an MDS, using Azuma-Hoeffding we show that with probability at least $1-\delta/2$
    \[
        \sum_{k,h} m_{hk} \leq 2H\sqrt{KH \log(2/\delta)}
    \]
    Show that the regret is upper bounded with probability $1-\delta$ by
    \[
        R(T) \leq 2\sum_{kh} b_{hk}(s_{hk},a_{hk}) + 2H\sqrt{KH \log(2/\delta)}
    \]

    \item Finally, we have that~\citep{DominguesMPKV21}
    \begin{align*}
        \sum_{h,k} \frac{1}{\sqrt{N_{hk}(s_{hk},a_{hk})}} \lesssim H^2S^2A +  2\sum_{h=1}^H\sum_{s,a} \sqrt{N_{hK}(s,a)}
    \end{align*}
    Complete this by showing an upper-bound of $H\sqrt{SAK}$, which leads to $R(T) \lesssim H^2S\sqrt{AK}$
\end{itemize}

\begin{algorithm}[t]
    \DontPrintSemicolon
    Initialize $Q_{h1}(s,a) = 0$ for all $(s,a) \in S\times A$ and $h=1,\ldots,H$
	\vspace{0.1in}

	\For{$k=1, \ldots, K$}{
		Observe initial state $s_{1k}$ \textit{(arbitrary)}

            Estimate empirical MDP $\wh{M}_k = (S, A, \wh{p}_{hk}, \wh{r}_{hk}, H)$ from $\mathcal{D}_{k}$
            {\scriptsize
                    \begin{align*}
                            \wh{p}_{hk}(s'|s,a) = \frac{\sum_{i=1}^{k-1} \mathds{1}\{(s_{hi},a_{hi},s_{h+1,i})=(s,a,s')\}}{N_{hk}(s,a)} , \quad
                            \wh{r}_{hk}(s,a)    = \frac{\sum_{i=1}^{k-1} r_{hi} \cdot \mathds{1}\{(s_{hi},a_{hi})=(s,a)\}}{N_{hk}(s,a)}
                    \end{align*}
            }

            Planning (by backward induction) for $\pi_{hk}$ using $\wh{M}_k$\;
            \For{$h=H,\ldots,1$}{
                $Q_{h,k}(s,a) = \wh{r}_{h,k}(s,a) + b_{h,k}(s,a) + \sum_{s'} \wh{p}_{h,k}(s'|s,a) V_{h+1,k}(s')$\;
                $V_{h,k}(s) = \min\{H, \max_a Q_{h,k}(s,a)\}$\;
            }
            Define $\pi_{h,k}(s) = \argmax_{a} Q_{h,k}(s,a)$, $\forall s,h$\;

		\For{$h=1, \ldots, H$}{
			Execute $a_{hk} = \pi_{hk}(s_{hk})$\;
            Observe $r_{hk}$ and $s_{h+1,k}$\;
            $N_{h,k+1}(s_{hk},a_{hk}) = N_{h,k}(s_{hk},a_{hk}) + 1$
		}
    }
    \caption{UCBVI}
    \label{alg:ucbvi}
\end{algorithm}

\appendix
\section{Weissmain inequality}
Denote by $\wh{p}(\cdot|s,a)$ the estimated transition probability build using $n$ samples drawn from $p(\cdot|s,a)$. Then we have that 
\[
    \mathbb{P}(\|\wh{p}_h(\cdot|s,a) - p_h(\cdot|s,a)\|_1 \geq \epsilon) \leq  (2^S - 2) \exp\Big(- \frac{n \epsilon^2}{2} \Big) 
\]

\bibliographystyle{plainnat}
\bibliography{bibliography}
\end{document}
